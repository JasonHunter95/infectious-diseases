\documentclass[11pt]{article}
\setlength{\oddsidemargin}{0in}
\setlength{\evensidemargin}{0in}
\setlength{\textwidth}{6.5in}

\usepackage{fancyhdr}
\pagestyle{fancy}
\usepackage{amsmath,amsfonts,amssymb}
\usepackage{epsfig}
\usepackage{subfigure}
\usepackage{placeins}
\usepackage{amsmath}
\usepackage[usenames,dvipsnames,svgnames,table]{xcolor}
\usepackage{amssymb}
\usepackage{setspace}
\usepackage{graphicx} % Include figure files
\usepackage{times}
\usepackage{amsthm}
\usepackage{hyperref}
\usepackage{enumitem}
\hypersetup{bookmarks=true, unicode=false, pdftoolbar=true, pdfmenubar=true, pdffitwindow=false, pdfstartview={FitH}, pdfcreator={Daniel Larremore}, pdfproducer={Daniel Larremore}, pdfkeywords={} {} {}, pdfnewwindow=true, colorlinks=true, linkcolor=blue, citecolor=Green, filecolor=magenta, urlcolor=cyan,}
\usepackage[parfill]{parskip}

% This little bit tells LaTeX where to look for figures.
% As written, it says to look first in Notes/figs_python, and then look in the current directory ./  
\graphicspath{{../Notes/figs_python/}{./}}

\begin{document}

\lhead{{\bf Mathematical \& Computational Modeling of Infectious Diseases \\ 
Homework 1}}
\rhead{{\bf D.B.\ Larremore\\2025}}
\renewcommand{\headrulewidth}{0.4pt}

{\bf Instructions:} 
\begin{itemize}[itemsep=-7pt]
	\item Please turn in a single PDF file.
	\item Please share a link to your code, but do not attach the actual code. 
	\item {\color{blue} By the way, if you don't have a github, why not make one? It's easy to do, and a great place to stash your code and link to it. Pop by office hours or phone a friend if you need help. It's never too early (or late) to learn!} 
	\item Handwritten math (scanned and included in a PDF) is fine, but please avoid 10MB+ file sizes! Reduce your image quality as needed. 
	\item Don't forget to list anyone you collaborated with. 
\end{itemize}
\vspace{0.1in}\hrule

\begin{enumerate}

\item The goal of this problem is to get you over any barriers with (i) getting Python set up, (ii), getting the SIR model implemented in a Forward Euler solver, and (iii) getting matplotlib set up.
	
	Write a function in Python that uses the Forward Euler method to simulate the SIR model {\it in which the population is slowly growing}.\footnote{Not to be turned in, but consider checking your work by first reproducing the three plots from Figure 1 of the Week 2 lecture notes in the absence of population growth. The parameters are: $N=1000$, $I_0=1$, $S_0=999$, with $\beta=1$, $\gamma=0.5$, then $\beta = 1.5$, $\gamma=0.5$, and finally $\beta = 2$, $\gamma=0.5$}
	
	In your simulations, suppose that we have an initial population of size $N=1000$, with $I_0=1$, $S_0=999$. Suppose that $\beta=1$, $\gamma=0.5$, and then $\mu_{\text{birth}}=0.01$ and $\mu_{\text{death}} = \tfrac{1}{2} \mu_\text{birth}$. 
	
	Produce a single plot that captures the dynamics of transmission over the appropriate amount of time for the population to grow by 50\% to a total of $N=1500$. Be sure to include a legend and your name in the title of the plot.
	
\clearpage
\item The goal of this problem is to show an important fact about transition rates in compartmental models. It is also a good chance to become refreshed on simple ODE solving and separation of variables. Finally, it makes good on a promise made in lecture notes to ask this homework question!
	
	Imagine that we are interested in SIR dynamics, but everyone starts out either infected or recovered, and no one starts out susceptible.
	\begin{enumerate}[label=\alph*.]
		\item Use this information to simplify the typical equation for $\dot{I}$.
		\item Solve your simplified differential equation with the initial condition $I(0) = I_0$.
		\item Manipulate your solution to derive the fraction of the initially infected people who are still infected at time $t$.
		\item Discuss this equation. What does it do over time? How is it related to the fraction of infected people who have {\it left} the infected class?
		\item This formula produces values between $0$ and $1$, and it tells us the probability that a randomly chosen infected person is still infected at time $t$. How does this relate to the cumulative distribution function (CDF) that describes the probability that someone is infected for less than or equal to $t$ units of time? Take a derivative of the CDF to get a PDF for the duration of infection lengths is. Then, find out what this famous probability distribution is called, and write down its expected value.
		\item Use your results to explain how the recovery rate $\gamma$ is related to the typical amount of time a person remains infectious.
	\end{enumerate}

\clearpage
\item The goal of this problem is to (i) figure out how to solve the final epidemic size equation, and (ii) test the equation's predictions.

	\begin{enumerate}[label=\alph*.]
		\item First, explain how an epidemic's total size, also called its cumulative incidence, is related to $s_\infty$ and $r_\infty$. 
		\item Recall that 
		\begin{equation}r_\infty = 1-e^{-R_0 r_\infty}\label{myeq}\end{equation} 
		Though we can't solve this equation, we can use a valuable graphical technique as follows. First, we define a function $f(r_\infty) = r_\infty$ and define a second function $g(r_\infty) = 1-e^{-R_0 r_\infty}$. Note that $f$ is the left hand side of Eq.~\eqref{myeq}, while $g$ is the right hand side. This allows us to plot $f$ and plot $g$, both as a function of $r_\infty$, to see where they cross---that is, where $f=g$. 
		
		Create four plots for $R_0 \in \{0.9, 1.0, 1.1, 1.2\}$ with $f$ in black and $g$ in red. Use the {\bf fsolve} function to find the intersection point, and use matplotlib's {\bf scatter} function to plot a blue circle at the intersection.\footnote{Hint, use \href{https://docs.scipy.org/doc/scipy/reference/generated/scipy.optimize.fsolve.html}{the fsolve docs}, and note that, because fsolve wants to find roots (points where a function is zero), you can create an auxiliary function $h(r_\infty) = f(r_\infty)-g(r_\infty)$ which will be equal to zero at the point where $f$ and $g$ intersect!}
		\item Comment on what you see in the plots in the context of what we have learned about $R_0$. What do you see in your figures? What happens when $R_0 < 1$ ?
		\item Finally, test the predictions made by this final-size equation by using your SIR code and $\beta=1$, $\gamma=0.5$ by creating a new version of that epidemic with a green dotted line at the height of $r_\infty$. Does this final size prediction work?\footnote{Take care, as the units of our SIR plots and the units of our final size prediction are not the same! You may have to do a conversion...}
	\end{enumerate}
	

\clearpage
\item (Grad / EC): In class, we showed that the SIR model's disease-free equilibrium is stable when $s<\tfrac{1}{R_0}$ and unstable otherwise. Using $N=10^6$, and $\varepsilon = \frac{1}{N}$ as your perturbation, produce a single figure {\it using your simulation code and its output} that illustrates this point. Write a caption that explains the principle of stability, and explain how your figure illustrates it.\footnote{There are many possible ways to make such a figure, and write its caption! One could use all of the outputs of the simulation, for instance, but one could also use just those that illustrate the intended point.}

\end{enumerate}



\end{document}